Dans le cadre d’un processus de modernisation de la gestion des services publics marchands, l’état avait, depuis l’année 2006, promulgué la loi 54-05 relative à la gestion déléguée.  Cette loi visait à introduire le secteur privé dans la gestion des services publics locaux habituellement gérées par les communes et les établissement publics. Le transport urbain par autobus étant un service public local marchand fut parmi les premier à être géré en gestion déléguée. En effet, avant même la promulgation de la loi 54-06, des sociétés privées avaient signé des conventions de cogestion avec les communes pour la gestion service public de transport urbain par autobus (Marrakech en 2001, Meknès en 2005, etc.).
l’objectif fondamental de l’introduction du privé dans le secteur est de drainer des investissements supplémentaires  et moderniser la gestion étant donné que le système classique de régie communale ou autonome a atteint ses limites.
Actuellement, on récence une quarantaine de contrats de gestion dans le secteur couvrant pratiquement les métropoles et les villes moyennes du Royaume. Le milieu rural est également desservi par l’élargissement de certains périmètres de gestion à la périphérie des villes lorsque des communes rurale en formule le besoin. Le suivi par les commune appelées autorités délégantes en contrat. Cette prérogative est essentielle dans la mesure où elle permet de préserver l’équilibre économique du service et du pouvoir d’achat des usagers. Seulement, pour pouvoir assurer cette tache en temps réelle, les indicateurs contractuels rapportés par les délégataires (société privées chargées de la gestion) devront être vérifiés par autorité délégante et comparés au standards en vigueur.  C’est ainsi que le développement d’une plateforme objet du présent travail constitue un besoin pertinent et opportun pour mieux mener le suivi en question. Ces indicateurs seront traités au niveau de cet outil, traités automatiquement en vue de dégager des ratios utilisables par ailleurs à des fins de benchmarking.
Le présent rapport est organisé comme suit : le chapitre I est dédié à la présentation de l’organisme d’accueil et du client. Le chapitre II traite l’analyse du problème ainsi que la spécification des besoins. Le chapitre III présente l´étude conceptuelle et enfin le chapitre IV traite la partie réalisation.
