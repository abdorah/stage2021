Dans ce chapitre j'entame le projet dans son cadre général : présentation
de l’organisme d’accueil \textbf{UIR},
et de client, présentation de l'idée de l'application, présentation du
contexte du projet, de la problématique et des objectifs fixés,
et de la démarche de la réalisation que j'ai adopté.
\section{Présentation de l’organisme d’accueil}
\subsection{UIR}

L’Université Internationale de Rabat ou \textbf{UIR} est une université
privée fondée en 2010 sous contrat avec l’État marocain.
L'\textbf{UIR} concrétise ainsi le premier partenariat public-privé dans le
domaine de l'enseignement supérieur au Maroc.
Poursuivant l'objectif d'accompagner le Royaume du Maroc dans son
développement, l'\textbf{UIR} a développé un catalogue
de formation de haut niveau en adéquation avec les différentes stratégies
impulsées par le Maroc (Plan solaire marocain
, plan d'accélération industrielle, plan de digitalisation, etc.). Surtout
sur le plan informatique, l'\textbf{UIR} demeure
un partenaire important de l'état. Elle a plusieurs contributions en terme
d'éducation et de préparation des cadres, ainsi
qu'en terme de consulting, de recherche, de résolutions des problèmes
émergents, et en terme d’innovation.
\begin{figure}[H]
	\begin{center}
		\fbox{\includegraphics[scale=0.09]{logo-uir.jpg}}
		\caption{UIR (Organisme d’accueil)}
	\end{center}
\end{figure}
\subsection{Partenaire de consulting technique: Dyn IT}

\textbf{DYN IT MAROC} est le résultat de l’expérience de plusieurs
consultants et éducateurs en IT.
Elle propose des services en Ingénierie Informatique.
Sa mission est de vous faire collaborer efficacement. Elle s'engage à aider leurs clients
en fournissant des solutions de collaboration flexibles, évolutives et
surtout abordables.
Elle offrent des services de consultation et de formation en technologies
Microsoft. En fait, \textbf{DYN IT} est un partenaire Microsoft.
\begin{figure}[H]
	\begin{center}
		\fbox{\includegraphics[scale=0.6]{dynit.png}}
		\caption{DYN IT (Partenaire de consulting technique)}
	\end{center}
\end{figure}
\section{Présentation du Client}

La Direction Générale des Collectivités Territoriales \textbf{DGCT} est chargée de l’accompagnement des collectivités locales et des instances qui en relèvent dans l’exécution des missions de développement local qui lui  sont conférées en vertu des textes législatifs et réglementaires relatifs aux collectivités
territoriales. Cet accompagnement est d’ordre juridique, technique et financier. \textbf{DGCT} est chargée également, en
coordination avec les départements et organismes concernés, de
concourir au développement territorial. Bref, ses missions sont en gros:
\begin{itemize}
	\item[•] Planification et développement territorial.
	\item[•] Assistance des réseaux publics locaux, et des institutions
	      locales.
	\item[•] Suivi juridique et gestion des services locaux.
	\item[•] Amélioration de la mobilité urbaine et du transport.
	\item[•] Développement des compétences et transformation digitale.
	\item[•] Accompagnement financier des collectivités territoriales.
	\item[•] Coopération décentralisée.
\end{itemize}
\begin{figure}[H]
	\begin{center}
		\fbox{\includegraphics[scale=0.27]{logo-fr.png}}
		\caption{La DGCT (Client)}
	\end{center}
\end{figure}
\newpage
\section{Problématique}

On recense environ 40 conventions de cogestion et de contrats de gestion déléguée du transport urbain par autobus. Ces contrats fixent les droits et les obligations des parties prenantes en vue de garantir un service optimum à l’usager. Le volet du suivi revient de fait au communes délégantes en raison de leur responsabilité vis-à-vis des usagers électeurs.

Pour assurer le suivi, les délégataires en tant qu’opérateurs du service, devront transmettre à l’autorité délégante, de manière périodique, des données se rapportant aux aspects technique, commercial, financier, social et environnemental relatant les résultats de la gestion. Au vu de leur diversité et de leur quantité, le traitement et l’analyse de ces données nécessitent l’utilisation d’outils informatiques permettant la fiabilité au niveau des moyens et l’efficacité au niveau des conclusions auxquelles elles débouchent. A travers le développement d’une application informatique de traitement et de consolidation des données en question, j’espère contribuer à mettre à la disposition de la \textbf{DGCT} un outil en mesure de surmonter cette problématique et approcher les objectifs escomptés.

\section{Solution et Objectifs du projet}

La première étape de la conduite de projet est sans doute la plus évidente,
mais peut-être
aussi la plus cruciale. En effet, sans objectifs bien définis, il est
difficile de savoir où votre projet
va vous mener. Pour cela, avant de commencer dans la conception et la
réalisation de ce projet,
il faut tout d’abord fixer l’objectif principal de manière à développer la
ligne d’actions à mener.
Je cherche à concevoir et construire une application Web qui permet de
saisir, d'enregistrer,
et de traiter les données qui concernent les contrats de gestion de
transport. La
solution proposée est composée de deux partie: une première partie qui
concerne la
saisi et l'enregistrement des données sous la forme d'une plateforme web
liée à une base de données, et une deuxième partie qui concerne l'analyse des données.
Bref, je cherche une solution pour:
\begin{itemize}
	\item[•] Centraliser l’information contenue dans les contrats de gestion
	      déléguée et
	      leurs avenants le cas échéant.
	\item[•] Centraliser les documents exigés par le contrat que l’opérateur
	      est tenu de
	      fournir périodiquement à l’autorité délégante.
	\item[•] Permettre aux utilisateurs de partager en interne des
	      informations et des
	      documents.
	\item[•] Assurer le suivi de la gestion des contrats de transport urbain.
\end{itemize}
\section{Récapitulatif}
Dans ce chapitre introductif, j'ai pu décrire le conteste général du
projet, et
déterminer son objectif principal. En addition, les aspects que je vais
avoir besoin lors de la réalisation de cette application.
Le chapitre qui suit consiste la phase d'analyse des besoins du projet.
