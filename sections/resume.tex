Ce document représente une étude synthétique du travail réalisé dans le
cadre de mon stage d'été de deuxième année. J'ai effectué mon stage au sein de
l'université internationale de Rabat \textbf{UIR}  et de la Direction Générale des Collectivités Territoriales (Direction de la Mobilité Urbaine et du Transport) relevant du Ministère de l’Intérieur. Ce stage a duré les deux
mois de juillet et d'août de l'année 2021. L’objectif principal de ce travail
consiste à développer une plateforme Web pour la gestion et le suivi des
contrats de gestion déléguée du transport urbain par autobus. La réalisation du projet a été effectuée en trois étapes principales:
\begin{itemize}
	\item[•] analyse détaillée des besoins du client;
	\item[•] conception des différentes composantes de l'application;
	\item[•] enfin l'implémentation des objectifs soulignés.
\end{itemize}
J'ai donc eu la chance d'utiliser plusieurs technologies, outils et concepts qui m'ont aidé à développer cette solution, dont plusieurs cours que j'ai appris au cours de l'année précédente, et certains outils que j'ai eu la chance de découvrir tout au long du stage (vous trouverez dans les annexes de ce document).

Le présent rapport est organisé comme suit : le chapitre I est dédié à
la présentation de
l’organisme d’accueil et du client. Quant au chapitre II, il traite
l’analyse du problème ainsi que la spécification des besoins. Le
chapitre III présente l´étude conceptuelle, et enfin le chapitre IV traite la
partie réalisation.

\textbf	{\\Mots clés :\\ Gestion, Suivi, Donnée, Autobus, Gestion  délégués,
	Transport urbain, Centralisation, Application low-code.}
