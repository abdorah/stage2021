\documentclass[a4paper]{report}
\usepackage[T1]{fontenc}
\usepackage[french]{babel}
\usepackage{setspace}
\usepackage{graphicx}
\usepackage{wrapfig}
\usepackage{lscape}
\usepackage{bookmark}
\usepackage{rotating}
\usepackage{subcaption}
\usepackage{float}
\usepackage{eurosym}
\usepackage[headheight=13pt,top=3cm, bottom=2cm, left=1cm, right=1cm]{geometry}
\usepackage{pdfpages}
\usepackage{fancyhdr}
\pagestyle{fancy}
\usepackage{hyperref}
\hypersetup{
    colorlinks=true,
    linkcolor=black,
    citecolor=black,
    filecolor=black,
    urlcolor=black,
}
\usepackage{array,multirow,makecell}
\setcellgapes{3pt}
\setlength{\parindent}{1.1cm}
\makegapedcells
\newcolumntype{L}[1]{>{\raggedright\arraybackslash}b{#1}}
\newcolumntype{R}[1]{>{\raggedleft\arraybackslash}b{#1}}
\newcolumntype{C}[1]{>{\centering\arraybackslash}b{#1}}
\renewcommand{\footrulewidth}{1pt}
\fancyfoot[L]{ENSIAS}
\fancyfoot[C]{\textbf{\thepage}}
\fancyfoot[R]{Année universitaire: 2020/2021}
\begin{document}
\begin{titlepage}
	\begin{center}
		\begin{figure}[!h]
			\vspace{- 2 cm}
			\hspace{ 0 cm}
			\includegraphics[width=9em]{images/ensias.jpeg}
		\end{figure}
		\begin{figure}[!h]
			\vspace{- 3.34cm}
			\hspace{14 cm}
			\includegraphics[width=10em]{images/um5.jpeg}
		\end{figure}
	\end{center}

	\begin{center}
		\noindent \hspace{ 0.3 cm }\Huge \textbf{Rapport de Stage Fin de
			deuxième Année}
		\vspace*{0.1cm}

		\vspace*{0.1cm}
		\begin{center}
			\rule{0.9\linewidth}{1pt}
		\end{center}
		\begin{center}
			\noindent \hspace{ 0.4 cm}{\large \textsc{ Création d'une
					plateforme Web pour la gestion et le suivi des contrats du transport
					délégués}}\\
		\end{center}

		\vspace*{0.5cm} \noindent \hspace{ -0.5 cm} \large
		\begin{figure}[H]
			\begin{center}
				\includegraphics[width=3.5in,height=3in]{images/logo-uir2.jpg}
			\end{center}
		\end{figure}
		\raggedright
		{\textbf{\emph{Préparé par: KOTBI Abderrahamane}}}
		\raggedleft\rule{0.7\linewidth}{2pt}\\
		\raggedleft{\textbf{\emph{Membre Jury: *****}}}\\
		\Large\emph{Année universitaire: 2020-2021}

	\end{center}
\end{titlepage}
\pagenumbering{arabic} \setcounter{page}{1}
\addcontentsline{toc}{chapter}{Remerciements}
\begin{doublespace}
	\begin{center}
		\vspace*{1cm}

		\textbf{\huge{Remerciements}}

	\end{center}

	\textit{
		Avant d'aborder la description des parties importantes du projet,
		j'aimerai tout d’abord exprimer ma gratitude
		intense à toute personne qui a contribué énormément dans l'élaboration
		et la réalisation de ce travail. Je commence
		ainsi par offrir mes remerciements à l'intégralité des personnes
		travaillant au sein de l’école Nationale supérieure d’informatique et d’analyse
		des systèmes. Ces bonnes hommes et femmes qui ne cessent de nous préparer, moi
		et
		mes collègues, pour de telles expériences afin de bien s’intégrer dans
		le cadre réelle du marché de travail.
		Toute ma gratitude et profonde reconnaissance s’adressent à à tous mes
		encadrants de stages: Mme.\textbf{***f***} ****,
		M..\textbf{***m***} ****   . Je vous remercie énormément pour m'avoir
		garanti un encadrement de qualité pour bien mener
		et assurer la réalisation de ce travail dans les meilleures des
		conditions.
	}


	\clearpage
	\addcontentsline{toc}{chapter}{Résumé}
	\begin{center}
		\vspace*{1cm}

		\textbf{\huge{Résumé}}

	\end{center}
	Ce document représente une étude synthétique du travail réalisé dans le
	cadre de mon stage d'été de deuxième année. J'ai effectué mon stage au sein de
	l'université internationale de Rabat \textbf{UIR}. Ce dernier a duré les deux
	mois juillet et août de l'année 2021. L’objectif principal de ce travail
	consiste à développer une plateforme Web pour la gestion et le suivi des
	contrats du transport délégués. D'ailleurs, le travail effectué dans le cadre
	de ce projet m'a représenté la bonne chance pour appliquer mes savoirs et
	savoirs-faire acquis durant cette année. En ce sens, la réalisation du projet a
	passé en trois étapes principal: l'élaboration d'une analyse détaillée des
	besoins du client, conception des différentes composantes de l'application, et
	enfin l'implémentation des objectifs soulignés. Par conséquent, j'ai eu la
	chance d'utiliser plusieurs technologies, outils et concepts qui m'ont aidé à
	développer cette solution.

	Le reste de ce rapport est organisé comme suit : le chapitre I est dédié à
	la présentation de
	l’organisme d’accueil et du client. Quant au chapitre II, il traite
	l’analyse du problème ainsi que la spécification des besoins. Ensuite le
	chapitre III présente l´étude conceptuelle. Enfin le chapitre IV traite la
	partie réalisation.

	\textbf	{\\Mots clés :\\ Gestion, Suivi, Donnée, Contrats délégués,
		Transport urbain, Centralisation, Application low-code.}

	\clearpage
	\addcontentsline{toc}{chapter}{Abstract}
	\begin{center}
		\vspace*{1cm}
		\textbf{\huge{Abstract}}
	\end{center}

	This document represents a synthetic study of the work carried out as part
	of my second-year summer intern-ship.
	I did my intern-ship at the international university of Rabat \textbf{UIR}.
	The latter lasted the two months
	July and August of the year 2021. It aims to develop a web platform for the
	management and monitoring of
	delegated transport contracts. Moreover, the work carried out in this
	project represented good opportunity
	for me to apply and sharpen my skills and my knowledge acquired during this
	year. Thus, the realization
	of the project has gone through three main stages: the extraction of a
	detailed analysis of the client's
	needs, the design of the application's different components, and finally
	the implementation of the outlined objectives.
	Therefore, I had the chance to use several technologies, tools, and
	concepts that helped me develop this solution.

	\textbf{\\Keywords:\\Management, Monitoring, Data, Delegated contracts,
		Urban transport, Centralization, Low-code application.}
\end{doublespace}

\newpage

\renewcommand{\contentsname}{Table de matières}
\setcounter{tocdepth}{4}
\tableofcontents

\cleardoublepage

\addcontentsline{toc}{chapter}{\listfigurename}
\listoffigures
\newpage
\begin{doublespace}
	\addcontentsline{toc}{chapter}{Introduction générale}
	\begin{center}
		\vspace*{1cm}
		\textbf{\huge{Introduction générale}}
	\end{center}
	\renewcommand{\headrulewidth}{1pt}
	\fancyhead[R]{\textbf{Introduction générale}}
	\fancyhead[L]{\hspace*{5cm}}
	Dans le cadre de la modernisation de la gestion des secteurs publics
	collectifs
	marocains, l’état a créé des contrats de gestion déléguées. Parmi ces
	dernières, les
	contrats de gestion de transport urbain. Placer le citoyen au cœur des
	politiques et
	services publics en constitue un objectif fondamentale. En conséquence, le
	contrat
	vise à assurer des services publics de qualité, accessibles à tous les
	citoyens sans discrimination, à des coûts maitrisés tenant compte du
	pouvoir
	d’achat. D'où  l'importance de la bonne gestion et le suivi de ces
	contrats, pour garder
	les droits et les attentes de la population.
	Par ailleurs, l’état a pris ce sens pour réaliser ses objectifs d’ouverture
	sur l’économie
	mondiale, le soutien des entreprises publiques, et l’amélioration de ses
	services.
	En parallèle avec ceci, l’état a essayé de gagner les défis de la
	régionalisation.
	Par conséquent, la gestion et le suivis des contrats de transport a devenu
	une tâche
	fondamentale. Or, effectuer cette tâche en utilisant la méthode classique,
	par plusieurs documents,
	ne représente plus une bonne solution, vue le nombre de lecture et de
	vérification
	des données importantes dans les documents nombreux. En plus, cela ne
	permet pas d’en
	tirer profits pour prendre les bons décisions facilement. Ainsi, je propose
	une
	solution simple pour la gestion et la représentation des données en
	question.
	\newpage
	\chapter{Cadre générale du projet}
	\renewcommand{\headrulewidth}{1pt}
	\fancyhead[R]{\textbf{Chapitre \thechapter: Cadre générale du projet}}
	\fancyhead[L]{\hspace*{5cm}}
	Dans ce chapitre j'entame le projet dans son cadre général : présentation
	de l’organisme d’accueil \textbf{UIR},
	et de client, présentation de l'idée de l'application, présentation du
	contexte du projet, de la problématique et des objectifs fixés,
	et de la démarche de la réalisation que j'ai adopté.
	\section{Présentation de l’organisme d’accueil}
	\subsection{UIR}

	L’Université Internationale de Rabat ou \textbf{UIR} est une université
	privée fondée en 2010 sous contrat avec l’État marocain.
	L'\textbf{UIR} concrétise ainsi le premier partenariat public-privé dans le
	domaine de l'enseignement supérieur au Maroc.
	Poursuivant l'objectif d'accompagner le Royaume du Maroc dans son
	développement, l'\textbf{UIR} a développé un catalogue
	de formation de haut niveau en adéquation avec les différentes stratégies
	impulsées par le Maroc (Plan solaire marocain
	, plan d'accélération industrielle, plan de digitalisation, etc.). Surtout
	sur le plan informatique, l'\textbf{UIR} demeure
	un partenaire important de l'état. Elle a plusieurs contributions en terme
	d'éducation et de préparation des cadres, ainsi
	qu'en terme de consulting, de recherche, de résolutions des problèmes
	émergents, et en terme d’innovation.
	\begin{figure}[H]
		\begin{center}
			\includegraphics[scale=0.1]{images/logo-uir.jpg}
			\caption{Organisme d'acceuil}
		\end{center}
	\end{figure}
	\subsection{Partenaire de consulting technique: Dyn IT}

	\textbf{DYN IT MAROC} est le résultat de l’expérience de plusieurs
	Consultants en IT et Education.
	Nous conseillons et proposons des services en Ingénierie Informatique.
	Notre mission est de
	vous faire collaborer efficacement. Elle s'engage à aider leurs clients
	en fournissant des solutions de collaboration flexibles, évolutives et
	surtout abordables.
	Elle offrent des services de consultation et de formation en technologies
	Microsoft. En fait, \textbf{DYN IT} est un partenaire Microsoft.
	\begin{figure}[H]
		\begin{center}
			\includegraphics[scale=0.6]{images/dynit.png}
			\caption{Organisme d'acceuil}
		\end{center}
	\end{figure}
	\section{Présentation du Client}

	La Direction Générale des Collectivités Territoriales \textbf{DGCT} est
	chargée de la préparation des décisions du ministre de l’Intérieur,
	dans le cadre des attributions qui lui sont conférées en vertu des textes
	législatifs et réglementaires relatifs aux collectivités
	territoriales, et du suivi de leur exécution. Elle assure également l’appui
	et l'accompagnement juridique, technique et financier
	des collectivités territoriales, des instances qui en relèvent, des
	établissements de coopération intercommunale et des groupements
	des collectivités territoriales. Elle est chargée également, en
	coordination avec les départements et organismes concernés, de
	concourir au développement territorial. Bref, ses missions sont en gros:
	\begin{itemize}
		\item Planification et développement territorial.
		\item Assistance des réseaux publics locaux, et des institutions
		      locales.
		\item Suivi juridique et gestion des services locaux.
		\item Amélioration de la mobilité urbaine et du transport.
		\item Développement des compétences et transformation digitale.
		\item Accompagnement financier des collectivités territoriales.
		\item Coopération décentralisée.
	\end{itemize}
	\begin{figure}[H]
		\begin{center}
			\includegraphics[scale=0.27]{images/logo-fr.png}
			\caption{Client}
		\end{center}
	\end{figure}
	\newpage
	\section{Problématique}

	Dans le cadre de la bon gestion de territoire et des aménagements publics,
	plusieurs
	contrats de gestion déléguée du transport urbain sont signés. Ces derniers
	comportent
	plusieurs actes qui forment un ensemble de règles à respecter entre un
	délégant qui
	sont une ou plusieurs communes et un délégataire qui est une société. En
	addition,
	des pénalités sont imposées en cas de non respect des règles posées dans le
	contrat.
	Par conséquent, le suivi de la réalisation et du respect de ses actes est
	un devoir
	juridique. D’où l'importance d'avoir une solution pertinente pour veiller
	sur le
	respect des conditions prédéfinis.
	\\\textbf{Alors, comment peut on améliore les opérations de suivis et de
		traitements des données en questions?}
	\section{Solution et Objectifs du projet}

	La première étape de la conduite de projet est sans doute la plus évidente,
	mais peut-être
	aussi la plus cruciale. En effet, sans objectifs bien définis, il est
	difficile de savoir où votre projet
	va vous mener. Pour cela, avant de commencer dans la conception et la
	réalisation de ce projet,
	il faut tout d’abord fixer l’objectif principal de manière à développer la
	ligne d’actions à mener.
	Je cherche à concevoir et construire une application Web qui permet de
	saisir, d'enregistrer,
	et de traiter les données qui concernent les contrats de gestion de
	transport. La
	solution proposée est composée de deux partie: une première partie qui
	concerne la
	saisi et l'enregistrement des données sous la forme d'une plateforme web
	liée à une
	base de donné, et une deuxième partie qui concerne l'analyse des données.
	Bref, je cherche une solution pour:
	\begin{itemize}
		\item Centraliser l’information contenue dans les contrats de gestion
		      déléguée et
		      leurs avenants le cas échéant.
		\item Centraliser les documents exigés par le contrat que l’opérateur
		      est tenu de
		      fournir périodiquement à l’autorité délégante.
		\item Permettre aux utilisateurs de partager en interne des
		      informations et des
		      documents.
		\item Assurer le suivi de la gestion des contrats de transport urbain.
	\end{itemize}
	\section{Récapitulatif}
	Dans ce chapitre introductif, j'ai pu décrire le conteste général du
	projet, et
	déterminer son objectif principal. En addition, les aspects que je vais
	avoir besoin lors de la réalisation de cette application.
	Le chapitre qui suit consiste la phase d'analyse des besoins du projet.
	\newpage
	\chapter{Analyse}
	\renewcommand{\headrulewidth}{1pt}
	\fancyhead[R]{\textbf{Chapitre \thechapter: Analyse et conception}}
	\fancyhead[L]{\hspace*{5cm}}

	Ce chapitre représente le point de départ de mon travail. Premièrement,
	j'analyse et
	je spécifie les besoins du projet. Ensuite, j'identifie les différents
	acteurs. Et enfin, je
	modélise le tout dans un diagramme des cas d’utilisation général qui
	sera notre file conducteur
	durant la prochaine phase.
	\section{Étude du projet}
	\subsection{Introduction à l'analyse}
	Certes, la gestion et le suivis des données en utilisant des solutions
	Web ne date pas d'hier. Mais,
	elle demeure une solution optimisant en terme de temps et de
	productivité. Surtout,
	dans un tel cas où les informations sont, à la fois, cruciales et
	nombreuses. Ainsi, la solution proposée premièrement par la \textbf{DGST} est
	d'avoir une plateforme Web pour faciliter la collecte,
	le traitement, la présentation, et la centralisation de l'information.
	Sur la même longueur d'onde, j'ai
	travaillé sur la solution proposée avec une équipe de la \textbf{DGST}
	pour apporter les bonnes
	fonctionnalités répondantes aux besoins de cette dernière.
	\subsection{Description générale du projet}
	D'un point de vue technique le projet est décomposé en trois partie. La
	première concernant la partie client dans laquelle l'utilisateur peut saisir
	les données. Ensuite, une base de données et enfin la partie reporting.
	\begin{figure}[H]
		\begin{center}
			\includegraphics[scale=0.2]{images/pre-descip-projet.png}
			\caption{Description préliminaire du projet}
		\end{center}
	\end{figure}
	\subsection{Étude de l’existant}
	Après la recherche et des exemples similaire à notre projet, on a
	trouvé une diversité des sites web et des applications dédiées à la gestion et
	au suivi.
	À titre d'exemple on peut considérer l'application \textbf{\large Ecan}:\\
	Généralement, les applications similaires sont des applications qui
	aide à la gestion de donnée et des ressources, et la prise des décisions.
	L’application \emph{Environnement Canterbury (ECan)} fait partie du
	gouvernement local de la région de Canterbury en Nouvelle-Zélande. Ils ont
	utilisé la plate-forme \emph{Power
		(PowerApps, Microsoft Flow et Power BI)} pour gérer et rendre compte
	efficacement des projets d'eau douce et de ressources naturelles.\\
	D’après l'article intitulé Environnement Canterbury accélère le suivi
	des
	résultats avec \emph{la Power Platform} :
	"Environnement Canterbury travaille en partenariat avec les communautés
	de
	Canterbury pour promouvoir la gestion durable des ressources
	naturelles. Cela
	implique l'utilisation de méthodes innovantes, rentables et
	techniquement
	excellentes, et garantit que la prise de décision est basée sur des
	informations
	de la plus haute qualité. Ils travaillent sur des programmes de
	résultats
	environnementaux à long terme qui consistent en plusieurs étapes et
	projets
	connexes. ECan avait besoin d'une solution abordable qui offrirait une
	plus
	grande cohérence entre les projets, des niveaux de visibilité plus
	élevés et un
	accès plus rapide aux données."
	\begin{figure}[H]
		\begin{center}
			\includegraphics[scale=0.32]{images/example.png}
			\caption{Exemple de Environnement Canterbury (ECan)}
		\end{center}
	\end{figure}
	\newpage
	\section{Spécification des besoins}

	\subsection{Spécification des acteurs}
	Vu que les informations gérées par cette application sont des informations confidentielles de l'état, l'application va avoir en général un seul acteur:
	L'administrateur est la personne chargée de charger les informations des différentes communes et de gérer leurs données, préparées par les différentes communes du Maroc.
	\subsection{Besoins fonctionnels}
	\begin{itemize}
		\item Gestion des délégants et des délégataires: L'administrateur doit avoir
		      une section où il pourra gérer les délégants et les délégataires.
		\item Gestion des contrats: Il doit avoir également la possibilité de gérer les contrats signés. En addition, il peut suivre les avenants et les révisions de chaque contrats. Ainsi que le status de chaque avenant ou révision.
		\item Gestion des Avancements: L'application doit permettre de gérer les avancements semestriels des contrats. Aussi, consulter et suivre les changements apportés à chaque période.
		\item Présentation des tableaux de bord: L'application doit présenter des données historisées de chaque contrats permettant de bien estimer la qualité de service en question et prendre les bonnes décisions.
	\end{itemize}
	\subsection{Les besoins non fonctionnels}
	Les besoins fonctionnels sont basiques pour un fonctionnement
	correcte et une réponse fiable aux besoins des utilisateurs, mais il y a des
	autres besoins qui tendent à améliorer la performance et la qualité de
	l'application pour une utilisation plus adéquate.
	\begin{itemize}
		\item Fiabilité de la plateforme: L’application doit
		      fonctionner sans erreur.
		\item Ergonomie, souplesse et confort d’utilisation: Pour
		      faciliter l’utilisation, notre plateforme doit offrir une interface unifiée,
		      conviviale et ergonomique.
		\item Gain de temps: L'application doit optimiser les
		      traitements pour avoir un temps de réponse minimale.
		\item Maintenabilité et sociabilité: La source de l'application doit être compréhensible
		      afin d’assurer son état évolutif et extensible par rapport aux besoins des utilisateurs. En outre, l’expérience des utilisateurs doit être meilleurs.
		\item Sécurité: Notre plateforme doit  être très authentique en ce qui concerne les informations confidentielles des communes.
	\end{itemize}
	\section{Analyse Fonctionnelle}
	Suivant à ce qui précède et d’après l’ensemble des documents communiqués par mes encadrants, j’ai essayé de créer une décomposition hiérarchique des fonctionnalités du projet.
	\begin{figure}[H]
		\begin{center}
			\includegraphics[scale=0.5]{images/WBS KPI.png}
			\caption{Décomposition fonctionnelle du projet}
		\end{center}
	\end{figure}
	Ainsi, J'ai divisé le projet en deux parties fondamentales. En plus, on peut voir clairement les grandes fonctionnalités demandées dans cette application. Par conséquent, l'application peut être divisée en deux partie: partie saisi et partie reporting.
	\begin{itemize}
		\item La partie saisi est la partie dans laquelle l'administrateur peut ajouter des délégants et des délégataires. En conséquences, il peut définir un contrat. Par ailleurs, le contrat définit peut avoir des mises à jour semestriels des informations qui le concerne.
		\item La partie reporting est la partie qui concerne les tableaux de bords de l’application. Il doit bien présenter les informations saisîtes. En plus, il est préférables d'avoir la possibilité de générer des documents contenants les informations saisîtes.
	\end{itemize}
	\section{Analyse technique}
	La solution que je propose est une solution basée sur les données.
	Autrement dite, elle permet à son utilisateur d'interagir avec plusieurs
	informations. En outre, elle représente une solution spécialisée pour
	l'acquisition, la gestion et la présentation d'informations. Ainsi, le groupe du projet à proposer l'ensemble des outils suivants:
	\begin{figure}[H]
		\begin{center}
			\includegraphics[scale=0.41]{images/outilsDB.png}
			\caption{Décomposition hiérarchique orientée livrable du projet}
		\end{center}
	\end{figure}
	Microsoft SQL Server est un système de gestion
	de base de données relationnelle développé
	par Microsoft. En tant que serveur de base de
	données, il s'agit d'un produit logiciel dont la
	fonction principale est de stocker et de
	récupérer des données à la demande d'autres
	applications logicielles, qui peuvent
	s'exécuter sur le même ordinateur ou sur un
	autre ordinateur via un réseau.
	\begin{figure}[H]
		\begin{center}
			\includegraphics[scale=0.41]{images/outilsDEV.png}
			\caption{Décomposition hiérarchique orientée livrable du projet}
		\end{center}
	\end{figure}
	Power Apps est un service permettant de créer
	et d'utiliser des applications professionnelles
	personnalisées qui se connectent à vos
	données et fonctionnent sur le Web et les
	appareils mobiles - sans le temps et les frais
	de développement de logiciels personnalisés.
	\begin{figure}[H]
		\begin{center}
			\includegraphics[scale=1.1]{images/outilsREP.png}
			\caption{Décomposition hiérarchique orientée livrable du projet}
		\end{center}
	\end{figure}
	Power BI est un service d'analyse
	commerciale de Microsoft. Il vise à fournir des
	visualisations interactives et des capacités de
	business intelligence avec une interface
	suffisamment simple pour que les utilisateurs
	finaux puissent créer leurs propres rapports et
	tableaux de bord. Il fait partie de la plate-
	forme Microsoft Power.\\
	Suite à cette proposition le consultant technique a approuver ces choix. Ainsi la décision est prise d'utiliser ces technologies Microsoft afin de réaliser une application Low-Code pour la gestion des contrats de transport délégués.
	\section{Récapitulatif}
	\section{Récapitulatif}
	Dans ce chapitre, nous avons déterminé les acteurs principaux dans notre projet ainsi que leur
	besoin. Ensuite, nous avons établi le Back log produit sur lequel nous nous sommes appuyés
	pour bâtir nos sprints. Cette étude sera notre base de travail dans le restant du chemin à
	savoir : la conception et la réalisation de notre projet. Dans le chapitre suivant nous allons
	exposer notre vue conceptuelle vis-à-vis du projet.
	\chapter{Conception}
	Après la prise en charge de la question, j'ai abordé le
	principe de la création d'un site e-commerce qui répond aux normes et aux
	modalités existantes dans les applications du commerce digitales, avec
	l’adaptation des nouvelles technique qui touchent la globalité de la solution,
	avec la possibilité d'une connexion à chaque tentative d'accès.
	\section{Cas d'utilisation globale}
	\begin{figure}[H]
		\begin{center}
			%  \includegraphics[scale=0.7]{images/pfa2 utilisation globale.png}
			\caption{Cas d'utilisation globale}
		\end{center}
	\end{figure}
	\newpage
	Dans cette figure on fait la segmentation des besoins
	principaux dont l'application doit répondre, on distingue les trois types des
	utilisateurs Administrateur, client et visiteur: \\Les deux premiers types sont
	les seuls à avoir des comptes avec authentification de soi, la fonctionnalité
	de l’administrateur se focalise sur tout ce qui est comparable à la
	gestion.Toutefois, le client peut effectuer des achats ainsi qu'il peut évaluer
	son expérience et ses produits. \\ L'application est aussi ouverte pour les
	visiteurs ordinaux, ils peuvent avoir un panier, consulter les produits à la
	phase de la création des comptes ils peuvent devenir des clients.

	\subsection{Cas d'utilisation de l'administrateur:}
	\begin{figure}[H]
		\begin{center}
			%  \includegraphics[scale=0.7]{images/admin cas.png}
			\caption{Cas d'utilisation de l'administrateur}
		\end{center}
	\end{figure}
	Dans cette figure on illustre le cas de l'utilisation de
	l'administrateur de façon plus détaillée, comme c'est déjà mentionné
	l'administrateur après son authentification, il prend la responsabilité de la
	gestion des clients, leurs transactions, ainsi qu'aux produits du magazins.Il
	fait la mise à jour, la suppression ou l'ajout des produits pour une gestion
	des produits plus fluide.\\ N'oubliant pas qu'il aura la possibilité de la
	consultation des informations des comptes clients leurs détails et leurs
	transactions effectuées.

	\section{Cas d'utilisation du client:}
	\begin{figure}[H]
		\begin{center}
			%  \includegraphics[scale=0.7]{images/client cas (1).png}
			\caption{Cas d'utilisation du client}
		\end{center}
	\end{figure}

	L'utilisateur client est est l'acteur principale qu'on vise
	lors de la conception de l'application, il a un compte personnel d'où il peut
	effectuer plusieurs fonctionnalités dès qu'il s'authentifie il n'est plus
	visiteur. Lors de la consultation de son compte il peut faire des modifications
	sur le plan de ses données, pour les achats cette fonctionnalité est triviale
	ainsi disponible selon les critères de chaque produit.\\D'ou il peut faire
	l'évalution et la consultation de chaque produit dans sa commande. \\ La
	communication au client et le service de la vente forment des traits principaux
	lors de l'achat et la vente, ici le client peut toujours rapporter ses
	interrogations et les abus d'utilisation à l'administrateur.
	\newpage
	\section{Cas d'utilisation visiteur:}
	\begin{figure}[H]
		\begin{center}
			%  \includegraphics[scale=0.7]{images/client cas (2).png}
			\caption{Cas d'utilisation du visiteur}
		\end{center}
	\end{figure}

	Le client avant qu'il obtient un compte à lui, il reste un
	visiteur ordinaire, avec diffèrents aspects d'utilisation: chaque visiteur aura
	un panier dont il pourra effectuer ses achats, consulter les produits qui
	existent dans son panier ainsi qu'il peut ajouter des produits dans ce
	panier.\\
	Il peut à tout moment contacter l'administrateur pour les
	renseignements, les modalités de paiement des commandes et de livraison colis.

	\subsection{Diagramme de séquence Authentfication
		Administrateur/ client:}
	\begin{figure}[H]
		\begin{center}
			%  \includegraphics[scale=0.6]{images/sec auth (2).png}
			\caption{Diagramme de séquence Authentfication
				Administrateur/ client}
		\end{center}
	\end{figure}
	Selon l'acteur, le système approuve les sessions
	d'utilisation des deux acteurs Administrateur ou client: Lors de
	l'authentification le système exige la validité des identifiants avec la
	vérification de la base de données, on approuve la connexion à la base de ces
	résultats.
	\subsection{Diagramme de séquence création d'un compte
		client:}
	\begin{figure}[H]
		\begin{center}
			%  \includegraphics[scale=0.6]{images/sec auth (3) (1).png}
			\caption{Diagramme de séquence création d'un compte
				client}
		\end{center}
	\end{figure}
	La séquence  qui précède décrit la création des comptes
	effectuée par les visiteurs pour devenir des clients. Le visiteur entre ses
	données dans la case convenable, si ces données sont qualifiées valides alors
	le compte est valable. Le visiteur est un client à ce moment.
	\section{Diagramme de séquence ajout au panier:}
	\begin{figure}[H]
		\begin{center}
			%  \includegraphics[scale=0.6]{images/sec auth (4).png}
			\caption{Diagramme de séquence ajout au panier}
		\end{center}
	\end{figure}
	Pendant chaque achat, les clients consultent les produits,
	font leurs recherches puis arrivent à choisir les produits qui leurs convient,
	d'où l'idée d'avoir un panier comme l'expérience au magazin réellement. \\Le
	client ajoute des produits à son panier, mais pour qu'il puisse accomplir son
	achat, l'application nécessite une phase d'enregistrement des données clients,
	la vérification des données si favorable permet au client de compléter son
	achat et d'accéder au paiement finalement.

	\section{Diagramme de séquence Ajouter ou Supprimer ou
	  modifier les catégories:}
	\begin{figure}[H]
		\begin{center}
			%  \includegraphics[scale=0.6]{images/sec auth (5).png}
			\caption{Diagramme de séquence modifications des
				catégories}
		\end{center}
	\end{figure}
	Pour assurer la bonne gestion des diffèrents produits sur
	notre plateforme, on a fait de sorte de classifier les produits sous des
	catégories, chacune contient un genre différent des produits:\\
	L'administrateur est l'acteur en charge de faire des telles modifications comme
	la suppression, l'ajout ou la mise à jour des catégories dont on inscrit les
	produits. L'administrateur s'authentifie au système, avec un retour de
	validation il effectue la modification, suppression ou ajout. Cette action est
	envoyée à la base de donnée pour un enregistrement des changements, la base de
	données traite ces actions le retour donne l'application de ces changements sur
	plateforme réelle.
	\section{Structure conceptuelle de l'application:}
	\begin{figure}[H]
		\begin{center}
			%  \includegraphics[scale=0.6]{images/Capp.PNG}
			\caption{Structure des catégories}
		\end{center}
	\end{figure}
	Pour bien expliquer la vision conceptuelle de l'application
	réelle , on consacré une phase de l'étude pour faire décider la structure de
	l'interface utilisateur  qui nécessite la conception des  sections suivants
	disponible lors de la phase d'acceuil.
	\begin{itemize}
		\item \textbf{Ordinateurs portables:} Où les acteurs
		      trouvent la condition exacte du produit (Nouveau ou non ).
		\item \textbf{Accessoires:} L'utilisateur peut savoir
		      le niveau du stockage de chaque outil ainsi que l'ergonomie de ces accessoires
		      les souris, claviers ainsi les cables.
		\item \textbf{Contact/Plus d'informations:} A partir de
		      celles-ci les acteurs en question peuvent contacter l'administrateur.
		\item \textbf{Login/Register:} C'est la phase de
		      l'authentification aux comptes.
		\item \textbf{Panier:} C'est l'entité destinée aux
		      clients lors de l'achat des produits.

	\end{itemize}

	\newpage
	\section{Récapitulatif:}
	Face aux conditions actuelles de la distanciation sociale
	et la fermeture des magazins et des espaces makro, cela a bien incarné l'idée
	de l'achat en ligne d'où on a eu l'idée de la création d’une application web de
	l'e-commerce spécialisée en vente des gadgets électronique. Cette partie aborde
	encore toutes les besoins des utilisateurs dont le site doit répondre : celles
	de l’administrateur, le client	et même du visiteur.

	Dans la première section on a essayé de bien analyser à
	fond les besoins primordiales des utilisateurs qu'on a spécifié lors de la
	phase de l'argumentation des besoins, où on a cité trois types d'acteurs avec
	les typologies de leurs besoins, sous forme plus graphique on a utilisé le
	langage UML pour décrire tout les cas d'utilisation possible à chaque
	utilisateur de façon globale puis  détaillée pour bien déterminer les limites
	d'action de chaque utilisateur.
	Ensuite on a décider de rendre les actions plus dynamiques,
	avec des diagrammes de séquence, qui font la description des interactions entre
	le système et les acteurs lors de chaque cas d'utilisation.

	\chapter{Réalisation:}
	\fancyhead[R]{\textbf{Chapitre \thechapter:Réalisation}}
	\fancyhead[L]{\hspace*{5cm}}


	\section{Technologies de travail:}
	Ce chapitre citera toutes les technologies et les
	outils utilisés pour la mise en œuvre de l'application Laptop+ .
	\subsection{Langage Python:}
	\begin{figure}[H]
		\raggedleft{
			%  \includegraphics[scale=0.3]{images/pythob.png}
		}
	\end{figure}
	Python est un langage de programmation largement
	utilisé, interprété, orienté objet et de haut niveau avec une sémantique
	dynamique, utilisé pour la programmation à usage général. Il a été créé par
	Guido van Rossum et sorti pour la première fois le 20 février 1991. C'est un
	langage simple et intuitif tout aussi puissant que ceux des grands
	concurrents,open source , pour que chacun puisse contribuer à son développement
	, il est aussi compréhensible
	adapté aux tâches quotidiennes , permettant des temps
	de développement courts.

	\newpage

	\subsection{Frame work Django:}
	\begin{figure}[H]
		\raggedleft{
			%  \includegraphics[scale=0.3]{images/django.png}
		}
	\end{figure}
	Django est un framework Web Python de haut niveau qui
	encourage un développement rapide et une conception propre et pragmatique.
	Construit par des développeurs expérimentés, il prend en charge une grande
	partie des tracas du développement Web, vous pouvez donc vous concentrer sur
	l'écriture de votre application sans avoir à réinventer la roue. C'est gratuit
	et open source.

	\subsection{L'éditeur de texte Atome:}
	\begin{figure}[H]
		\raggedleft{
			%  \includegraphics[scale=0.3]{images/atome.png}
		}
	\end{figure}
	Atom est un éditeur de texte et de code source gratuit
	et open-source pour macOS , Linux et Microsoft Windows avec prise en charge des
	plug-ins écrits en JavaScript et Git Control intégré , développé par GitHub .
	Atom est une application de bureau construite à l'aide des technologies Web .La
	plupart des packages d'extension ont des licences de logiciel libre et sont
	construits et maintenus par la communauté.
	\newpage
	\subsection{ Les graphiques Chart.Js:}
	\begin{figure}[H]
		\raggedleft{
			%  \includegraphics[scale=0.4]{images/cahrtjs.png}
		}
	\end{figure}
	Chart.js est une bibliothèque JavaScript open source
	gratuite pour la visualisation de données , qui prend en charge 8 types de
	graphiques : bar , line , area , pie ( donut ), bubble , radar , polar et
	scatter. Créé par le développeur Web  Nick Downie, il est maintenant maintenu
	par la communauté et est la deuxième bibliothèque de graphiques JS la plus
	populaire sur GitHub par le nombre d'étoiles après D3.js , considéré comme
	beaucoup plus facile à utiliser mais moinspersonnalisable que ce dernier.
	Chart.js est rendu dans un canevas HTML5 et est largement considéré comme l'une
	des meilleures bibliothèques de visualisation de données.
	\subsection{Databases Sqlite:}
	\begin{figure}[H]
		\raggedleft{
			%  \includegraphics[scale=0.4]{images/sqlite.jpg}
		}
	\end{figure}
	SQLite est une bibliothèque in-process qui implémente
	un moteur de base de données SQL transactionnel autonome, sans serveur, à
	configuration zéro . Le code de SQLite est dans le domaine public et est donc
	libre d'utilisation à toutes fins, commerciales ou privées.
	\newpage
	\section{L'application Web Laptp+ :}
	L'application Laptop+ est une application web
	e-commerce spécialisée dans la vente des ordinateurs portables et les
	accessoires en relation avec ces derniers. \\Laptop+ expose à la vue une
	plateforme d'achat est si innovante et puissante, son design relaxant et
	l’explicité de son contenu crée un espace compatible à tous les utilisateurs.

	\section{ Visualisation de la base de données:}
	\subsection{ Table des catégories: }
	\begin{figure}[H]
		%  \includegraphics[scale=0.5]{images/WhatsApp Image 2021-06-21 at 15.34.09.jpeg}
		\caption{Table des catégories}
	\end{figure}
	Cette figure donne l'aspect technique des catégories
	citées à l'étude
	fonctionnelle , chaque produit a son propre Id, le lien
	exact c'est à dire le slag ainsi qu'il appartient à une catégorie de produits
	spécifique.
	\subsection{ Table des produits: }
	\begin{figure}[H]

		%  \includegraphics[scale=0.5]{images/WhatsApp Image 2021-06-21 at 11.37.55.jpeg}
		\caption{Table des produits}
	\end{figure}
	Chaque produit s'enregistre dans la base des données
	suivant plusieurs spécifications, l'identifiant du produit, nom , prix ainsi
	que la présentation visuelle du produit. Avec les informations par rapport à la
	disponibilité dans le stock et la condition du produit.
	\subsection{ Table des paniers: }
	\begin{figure}[H]

		%  \includegraphics[scale=0.5]{images/WhatsApp Image 2021-06-21 at 11.37.22.jpeg}
		\caption{Table des paniers}
	\end{figure}
	Les paniers fonctionnent comme des paniers réelles, ce
	tableau fait la spécification de la quantité des produits dans le panier,  à
	l'aide des identificateurs panier et produits pour bien déterminer le
	positionnement des produits dans les paniers.
	\subsection{ Table des achats: }
	\begin{figure}[H]

		%  \includegraphics[scale=0.5]{images/WhatsApp Image 2021-06-21 at 11.36.25.jpeg}
		\caption{Table des achats}
	\end{figure}
	En ce qui concerne les ordres d'achats enregistrés,
	table  des achats fait l'inscription de tout les données concernant les ordres,
	elle désigne l'adresse mail du client son adresse domicile, la	date dont
	l'ordre a été effectué en plus du prix totale de l'ordre.
	\subsection{ Table clients: }
	\begin{figure}[H]

		%  \includegraphics[scale=0.5]{images/WhatsApp Image 2021-06-21 at 11.35.41 (1).jpeg}
		\caption{Table clients}
	\end{figure}
	La table client comme son nom l'indique contient tout
	les indicateurs du client son nom, la date de la création de son compte client
	ainsi que son Id-utilisateur.

	\section{Création des pages de l'application web:}
	\subsection{Page d'acceuil:}
	\begin{figure}[H]
		\begin{center}
			%  \includegraphics[scale=0.7]{images/a1.PNG}
			\caption{Page d'acceuil}
		\end{center}
	\end{figure}
	\newpage
	La page d'acceuil expose une selection des produits,
	avec leurs images réelles, chaque case contient un produit différent avec la
	possibilité de l'ajouter en panier. En haut de la page on trouve le logo
	significatif, la barre en haut donne la possibilité d'avoir un panier ou faire
	le login aux comptes.
	\begin{figure}[H]
		\begin{center}
			%  \includegraphics[scale=0.7]{images/a2.PNG}
			\caption{Page d'acceuil section aide }
		\end{center}
	\end{figure}
	La figure indique les conseils à suivre pour bien
	maintenir les produits, ainsi qu'elle offre aux utilisateurs la possibilité de
	contacter l'administrateur pour signaler leurs problèmes.
	\subsection{Page authentifiation adminstrateur:}
	\begin{figure}[H]
		\begin{center}
			%  \includegraphics[scale=0.3]{images/WhatsApp Image 2021-06-21 at 17.59.39 (2).jpeg}
			\caption{Page authentifiation adminstrateur}
		\end{center}
	\end{figure}
	L’authentification est une étape simple,
	l'administrateur insert les données déjà inscrites dans la base des données, il
	aura accès à son compte quand ses données sont valables.
	\subsection{Page authentification adminstrateur:}
	\begin{figure}[H]
		\begin{center}
			%  \includegraphics[scale=0.3]{images/WhatsApp Image 2021-06-21 at 17.59.39 (2).jpeg}
			\caption{Page authentification adminstrateur}
		\end{center}
	\end{figure}
	\subsection{Page espace adminstrateur:}
	\begin{figure}[H]
		\begin{center}
			%  \includegraphics[scale=0.3]{images/WhatsApp Image 2021-06-21 at 17.59.39 (4).jpeg}
			\caption{Page espace administrateur}
		\end{center}
	\end{figure}
	C'est l'espace administrateur où il pourra avoir
	l'historique de toutes les actions, les produits vendus, l'état des paniers et
	des ordres des clients.
	\subsection{Graphe Graphe nombre d'utilisateur:}

	\begin{figure}[H]
		\begin{center}
			%  \includegraphics[scale=0.3]{images/WhatsApp Image 2021-06-22 at 00.49.13.jpeg}
			\caption{Graphe nombre d'utilisateur}
		\end{center}
	\end{figure}
	Dans cette figure on peut savoir à tout moment
	l'efficience de l'application grace à ce type de graphe qui indique le taux
	d'utilisateurs qui utilisent l'application par jours.
	\subsection{Graphe prévision des ventes:}
	Pour une gestion plus fiable et moins couteuse on a
	choisit de faire une prévision d'avance qui sert à prévenir les achats qui
	peuvent être effectués à la période d'un an, en utilisant la méthode de
	regression linéaire, pour diminuer les pertes qui influencent le capitale et la
	satisfaction des clients.
	\begin{figure}[H]
		\begin{center}
			%  \includegraphics[scale=0.6]{images/WhatsApp Image 2021-06-22 at 01.31.37.jpeg}
			\caption{Graphe prévision des ventes}
		\end{center}
	\end{figure}
	\subsection{Page produit type PC Portable:}
	\begin{figure}[H]
		\begin{center}
			%  \includegraphics[scale=0.7]{images/T1.PNG}
			\caption{Page produit type PC Portable}
		\end{center}
	\end{figure}
	Cette page affiche tout les produits de type
	ordinateurs portables, en accordant le prix de chaque produit, sa condition et
	sa fiche informative pour mieux connaitre le produit.
	\subsection{Page produit type accessoire:}
	\begin{figure}[H]
		\begin{center}
			%  \includegraphics[scale=0.7]{images/T2.PNG}
			\caption{Page produit type accessoire:}
		\end{center}
	\end{figure}

	La plateforme offre une diversité des accessoires
	souris, cables de types qui diffèrent selon l'utilité, chaque case donne la
	description technique des accessoires ainsi que la possibilité de les ajouter
	au panier pour effectuer l'achat.

	\subsection{Page produit type stockages:}
	\begin{figure}[H]
		\begin{center}
			%  \includegraphics[scale=0.7]{images/T3.PNG}
			\caption{Page produit type stockage:}
		\end{center}
	\end{figure}

	Les accessoires du stockage font aussi parties des
	produits exposés avec leurs deux types SDD et HDD, les accessoires stokages
	sont réparties dans des cases avec le prix qui leurs correspond et l'image du
	produit.
	\subsection{Recapitulatif:}
	La réalisation du projet est un chapitre très
	descriptive et abrégeant toutes les étapes réelles de la mise en œuvre de
	l'application web. En premier, il aborde toutes les outils utilisés tel que le
	langage  Python. Ensuite, ce chapitre en son intégralité mentionne toutes
	technologies utisées pour la création des interfaces dynamiques. Finalement une
	grande partie, consacrée à la consultation de toutes les parties du site et de
	tous les cas de l’utilisation du « Laptop+ », avec l’insertion des screenshots
	qui simulent toutes les utilisations possibles avec la totalité des
	fonctionnalités offertes par l'application ainsi que celles du graphe de
	prévision qui sert à la bonne gestion des ventes au sein de l'application.
	\chapter{Conclusion générale et perspectives}
	\fancyhead[R]{\textbf{Chapitre \thechapter:Conclusion
			générale et perspectives}}
	\fancyhead[L]{\hspace*{5cm}}
	\section{Conclusion générale}
	L'e-commerce n'est pas la thématique qu'on aura
	apprendre seulement sous un contexte académique, mais un état d'esprit de mise
	en question et de prise de decision dans un cadre entreprenariale qui soutient
	la digitalisation de plusieurs secteurs. Cela explique notre choix de sujet su
	lequel on travaillait en particulier ces derniers mois, sous la demande énorme
	de l'informatisation des opérations d'achat effectuées par les consommateurs
	dans l'agenda des réglementations conçues sous le cadre de la pandémie.

	Actuellement, les documentations montrent que les
	achats en ligne sont plus utiles de la part des clients et pour les firmes en
	terme des revenus importants, ce qui les distingue comme les points que
	l'application vise en premier.

	Notre idée part d'un contexte digitale qui s'intitule
	\textbf{Laptop+}: C’est une application à utiliser par les entreprises
	spécialisées par la vente des ordinateurs, accessoires et des outils de
	stockage, pour encourager les client à utiliser ce mode d'achat, en leurs
	offrant une solution pratique, fiable et en plus  sécurisé en terme des
	transactions et des paiements.

	Ce projet a été une belle occasion pour travailler en
	groupe en des conditions qui ne sont pas assez simples, pour développer nos
	connaissances et nos compétences et surtout pour
	découvrir de nouvelles fonctionnalités des plateformes d'achats en ligne et des
	prévision des ventes et des langages tels que Python , la création des graphe
	dans la phase technique.
	\section{Perspectives}
\end{doublespace}
\chapter{Webographie}
\fancyhead[R]{\textbf{Chapitre
		\thechapter:Webographie}}
\fancyhead[L]{\hspace*{5cm}}
\section{Plateforme open source:}
\begin{itemize}
	\item \textbf{[kaggle]:}
	      https://www.kaggle.com/carrie1/ecommerce-data, dernière mise à jour le
	      15/06/2021.
	\item \textbf{[GitHub]:}
	      https://github.com/topics/ecommerce-website, dernière mise à jour le
	      20/06/2021.
\end{itemize}
\section{Cours:}
\begin{itemize}
	\item \textbf{Coursera:}
	      https://www.coursera.org/specializations/django, dernière mise à jour le
	      20/06/2021 mais avec inscription.

	\item\textbf{Udemy:}https://www.udemy.com/course/learn-python-django-from-scrat
	      ch/, dernière mise à jour 21/06/2021.
	\item \textbf{Citycours
		      :}https://sites.uclouvain.be/P2SINF/chartjs.html, dernière mise à jour le
	      21/06/2021.
\end{itemize}
\section{Dataset:}
\begin{itemize}
	\item \textbf{Kaggle:}
	      https://www.kaggle.com/jzeferino/predictive-sales-with-linear-regressions,
	      dernière mise à jour le\\ 19/06/2021.
\end{itemize}
\end{document}
