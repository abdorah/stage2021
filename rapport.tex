%% @Author: Ines Abdeljaoued Tej
%  @Date:   2018-06
%% @Class:  PFE de l'ESSAI - Universite de Carthage, Tunisie.

\documentclass[a4paper, oneside, 12pt, final]{extreport}
\usepackage{graphicx}

\parindent 0cm
\usepackage{makeidx}
\makeindex

\usepackage[lined,boxed,commentsnumbered, french, ruled,vlined,linesnumbered]{algorithm2e}
\usepackage{amsthm}
\newtheorem{theorem}{Theorem}[chapter]
\newtheorem{definition}{Definition}[chapter]
\newtheorem{exemple}{Example}[chapter]


%\usepackage[nottoc]{tocbibind}
%\addcontentsline{toc}{section}{References}

\providecommand{\keywords}[1]{\textbf{\textit{Mots clés---}} #1}
\providecommand{\keywordss}[1]{\textbf{\textit{Keywords---}} #1}

\usepackage{etoolbox}
%\makeatletter
%\patchcmd{\thebibliography}{%
%  \chapter*{\bibname}\@mkboth{\MakeUppercase\bibname}%{\MakeUppercase\bibname}}{%
%  \section{References}}{}{}
%\makeatother



\usepackage[nottoc]{tocbibind}

\textwidth 18cm
\textheight 24cm
\topmargin -0.5cm
\oddsidemargin -1cm

% set font encoding for PDFLaTeX or XeLaTeX
\usepackage{ifxetex}
\ifxetex
  \usepackage{fontspec}
\else
  \usepackage[T1]{fontenc}
  \usepackage[utf8]{inputenc}
  \usepackage{lmodern}
\fi


% Enable SageTeX to run SageMath code right inside this LaTeX file.
% documentation: http://mirrors.ctan.org/macros/latex/contrib/sagetex/sagetexpackage.pdf
%\usepackage{sagetex}


\newcommand{\reportTitle} {%
  %\textsc{Graduation Project Report}
  \textsc{Projet de Fin d'\'etudes}
}

\newcommand{\reportAuthor} {%
  FirstName \textsc{LastName}%
}

\newcommand{\reportSubject} {%
  My very attractive \\ Title%
}

\newcommand{\dateSoutenance} {%
  12/06/2018%
}

\newcommand{\studyDepartment} {%
  Entreprise d'accueil %Statistique
}

\newcommand{\ESSAI} {%
  %Higher School of Statistics and Information Analysis
  Ecole Sup\'erieure de la Statistique et de l'Analyse de l'Information
}

%\newcommand{\codePFE} {% Reference
%  Code PFE%
%}

\newcommand{\juryPresident} {%
  Mr Ben Foulen \textsc{Foulenia}%
}
\newcommand{\juryPresidentDesc} {%
  President%
}

\newcommand{\juryMemberOne} {%
  Ms Ben Foulena \textsc{Foulen}%
}
\newcommand{\juryMemberOneDesc} {%
  Examiner %Mentor
}

\newcommand{\juryMemberTwo} {%
  Mr Ben Foulen \textsc{Fouleni}%
}
\newcommand{\juryMemberTwoDesc} {%
  Reviewer% Examiner, Reporter
}

\newcommand{\juryMemberThree} {%
	M. Ben Foulen \textsc{Fouleni}%
}
\newcommand{\juryMemberThreeDesc} {%
	Supervisor% Examiner, Reporter
}

\newcommand{\juryMemberFour} {%
	M. Ben Foulen \textsc{Fouleni}%
}
\newcommand{\juryMemberFourDesc} {%
	Mentor% Examiner, Reporter
}


\newcommand{\specialcell}[1]{%
  \begin{tabularx}{\textwidth}{@{}X@{}}#1\end{tabularx}%
}

%%%%%%%%%%%%%%%%%%%%%%%%%%%%%%%%%%%%%%%%%%%%%%%%%%%%%%%
% Add your own commands here
%%%%%%%%%%%%%%%%%%%%%%%%%%%%%%%%%%%%%%%%%%%%%%%%%%%%%%%
\newcommand{\MyCommand} {%
  Does nothing really%
}


% used in maketitle
\title{\reportSubject}
\author{\reportAuthor}

% Enable SageTeX to run SageMath code right inside this LaTeX file.
% documentation: http://mirrors.ctan.org/macros/latex/contrib/sagetex/sagetexpackage.pdf
%\usepackage{sagetex}

%\hypersetup{
%  pdftitle={\reportTitle~-~\reportSubject},%
%  pdfauthor={\reportAuthor},%
%  pdfsubject={\reportSubject},%
%  pdfkeywords={report} {internship} {pfe} {enis}
%}

\usepackage{graphics}
\usepackage{graphicx}


\usepackage[acronym,toc,section=chapter]{glossaries}
\makeglossaries

\newacronym{abc}{ABC}{A contrived acronym}
\newacronym{efg}{EFG}{Another acronym}
\newacronym{svm}{SVM}{Support Vector Machines}

\pagenumbering{roman} 

\usepackage[utf8]{inputenc}
\usepackage[french]{babel}

\begin{document}
\thispagestyle{empty}
\begin{titlepage}
\begin{center}


%%%%%%%%%%%%%%%%%%%%%%%%%%%%%%%%%%%%%%%%%%%%%%%
% THE HEADER
%%%%%%%%%%%%%%%%%%%%%%%%%%%%%%%%%%%%%%%%%%%%%%%

\includegraphics[scale=0.15]{embleme.jpg}
\vspace{0.5cm}

{%
  \fontsize{9pt}{9pt}\selectfont%
  \begin{tabular}{c}
    R\'epublique Tunisienne \\
    Minist\`ere de l'Enseignement Supérieur et de la Recherche Scientifique \\%
    Universit\'e de Carthage - \ESSAI{}\\ 
  \end{tabular}
}

\vspace{0.5cm}

\includegraphics[scale=0.04]{universite-carthage.jpg}


%%%%%%%%%%%%%%%%%%%%%%%%%%%%%%%%%%%%%%%%%%%%%%%
% THE PAGE CONTENT
%%%%%%%%%%%%%%%%%%%%%%%%%%%%%%%%%%%%%%%%%%%%%%%

\vspace{5pt} {%
  \renewcommand*{\familydefault}{\defaultFont}
  \fontsize{46pt}{46pt}\selectfont%
  % MEMOIRE\\%
  %\reportTitle{}%\\\textsc{Report}\\%
}

%\vspace{5pt}

\vspace{10pt}
{\textit{Rapport de Projet de Fin d'Etudes soumis afin d'obtenir le}}\\

\vspace{10pt}
{\textbf{\large Diplôme National d'Ingénieur en Statistique et Analyse de l'Information}}\\

\includegraphics[scale=0.4]{logo-essai.jpg}\\

\vspace{5pt}
\textbf{\textit{Réalisé par}}\\
\vspace{10pt} {%
  \fontsize{14pt}{14pt}\selectfont%
  {\bfseries\Large\sc \reportAuthor}\\
}%

\vspace{5pt} {%
  \renewcommand*{\familydefault}{\defaultFont}
  \fontsize{27pt}{27pt}\selectfont%
  \rule{0.5\textwidth}{.4pt}\\
  \vspace{10pt}
  \reportSubject{}\\%
  \vspace{10pt}
  \rule{0.5\textwidth}{.4pt}
}

\vspace{5pt}
Soutenu le\, \dateSoutenance\,\, devant le Jury compos\'e de :\\
%Soutenu le \dateSoutenance, devant la commission d'examen:\\
\vspace{10pt}
\begin{tabular}{p{0.3\linewidth} p{0.15\linewidth}}
  \juryPresident{} & \juryPresidentDesc{}\\
  \juryMemberOne{} & \juryMemberOneDesc{}\\
  \juryMemberTwo{} & \juryMemberTwoDesc{}\\
  \juryMemberThree{} & \juryMemberThreeDesc{}\\
%  \juryMemberFour{} & \juryMemberFourDesc{}\\
\end{tabular}

%\vfill

\vspace{10pt}%
\textbf{\textit{Projet de Fin d'Etudes fait \`a}}\\
\vspace{5pt}
(\studyDepartment)\\
%\includegraphics[scale=0.4]{logo-studyDepartment.jpg}
\end{center}
\end{titlepage}

% ###############################
% # HELP COMMANDS               #
% ###############################
%
% -1 \part{part}
%  0 \chapter{chapter}
%  1 \section{section}
%  2 \subsection{subsection}
%  3 \subsubsection{subsubsection}
%  4 \paragraph{paragraph}
%  5 \subparagraph{subparagraph}


%%%%%%%%%%%%%%%%%%%%%%%%%%%%%%%%%%%%%%%%%%%%%%%%%%%%%%%
% Dédicace et Remerciements
%%%%%%%%%%%%%%%%%%%%%%%%%%%%%%%%%%%%%%%%%%%%%%%%%%%%%%%

%\chapter*{Dedication}
\chapter*{D\'edicace}
%\addcontentsline{toc}{chapter}{Dedication}
\thispagestyle{empty}
%
%For all they have endured to satisfy all my needs and wishes

\begin{center}
{\it 
	
A ... pour son(leur) sacrifice et son(leur) soutien, \\
en témoignage de mon infinie reconnaissance et mon profond attachement \\
\vspace{1cm}
A tous ceux qui me sont chers...

}
\end{center}
%
%\nopagebreak{%
% And maybe a quote here
% \raggedright\hspace{5.75cm} To all of you,~\\
%\raggedright\hspace{7.75cm} I dedicate this work.
%  \raggedleft\normalfont\large\itshape{} \reportAuthor\par%
%}
%
%\cleardoublepage%

%\chapter*{Thanks}
\chapter*{Remerciements}
%\addcontentsline{toc}{chapter}{Thanks}
\thispagestyle{empty}
%
%Au terme de ce travail (A l'issue de ce travail), je tiens à remercier M., Mme, Pr., Dr. pour sa disponibilité et ses conseils judicieux. \\

Je n'aurais jamais pu réaliser ce projet sans la précieuse aide et sans le soutien d'un grand nombre de personnes dont la générosité, la bonne humeur et l'intérêt manifestés à l'égard de mon PFE m'ont permis de progresser. \\

%En premier lieu, je tiens à remercier mon encadrant universitaire, \juryMemberFour{}, pour la confiance qu'il m'a accordée en acceptant d'encadrer ce travail, pour ses multiples conseils et pour toutes les heures qu'il a consacrées à diriger ce travail. \\ 

%Je souhaiterais exprimer ma gratitude à \juryMemberThree{}, pour m’avoir donné envie de réaliser un mémoire sur ... au sein de \og \studyDepartment \fg. Je le remercie également pour son accueil chaleureux à chaque fois que j'ai sollicité son aide, ainsi que pour ses multiples encouragements. J’ai été extrêmement sensible à ses qualités humaines d'écoute et de compréhension tout au long de ce travail de mémoire. \\

%J'aimerais également dire à \juryPresident{} à quel point je suis honorée pour avoir accepté de présider ce jury de PFE. Je suis infiniment gré à  \juryMemberOne{} de s’être rendu disponible et d’avoir accepté la fonction de rapporteur. De même, je suis particulièrement reconnaissant(e) à (\juryMemberTwo{} de l'intérêt qu'il/elle a manifesté à l'égard de ce projet en s'engageant à être rapporteur. \\

Ma reconnaissance va à ceux qui ont plus particulièrement assuré le soutien affectif de ce travail : ma famille ainsi que mes amis. Mes parents... 


This document represents a synthetic study of the work carried out in part of my second-year summer internship. I did my internship within
the international university of Rabat \ textbf {UIR} and the Directorate General of Territorial Communities (Directorate of Urban Mobility and Transport) under the Ministry of the Interior. This internship lasted both
the months of July and August of the year 2021. The main objective of this work
consists of developing a web platform for the management and monitoring of
contracts for the delegated management of urban bus transport. The project was carried out in three main stages:
\begin{itemize}
    \item[•] detailed analysis of customer needs;
    \item[•] design of the different components of the application;
    \item[•] finally the implementation of the underlined objectives.
\end{itemize}
So I had the chance to use several technologies, tools, and concepts that helped me develop this solution, including several courses that I learned during the previous year, and some tools that I had the chance to discover throughout the internship (you will find in the appendices of this document).

This report is organized as follows: Chapter I is dedicated to
the presentation of
the host organization and the client. Chapter II deals with
the analysis of the problem as well as the specification of the needs.
Chapter III presents the conceptual study, and finally chapter IV deals with the
realization part.

\textbf{\\Keywords:\\Management, Monitoring, Data, Bus, Delegated monitoring, contracts, Urban transport, Centralization, Low-code application.}

%%%%%%%%%%%%%%%%%%%%%%%%%%%%%%%%%%%%%%%%%%%%%%%%%%%%%%%
% Divers chapitres
%%%%%%%%%%%%%%%%%%%%%%%%%%%%%%%%%%%%%%%%%%%%%%%%%%%%%%%

\tableofcontents
%\addcontentsline{toc}{chapter}{\contentsname}

\listoffigures
%\addcontentsline{toc}{chapter}{Liste des Figures}
\listoftables
%\addcontentsline{toc}{chapter}{Liste des Tableaux}
\listofalgorithms
\addcontentsline{toc}{chapter}{Liste des algorithmes}
\cleardoublepage

\newpage
\pagenumbering{arabic}
\chapter*{Introduction}
\label{chap:general_intorduction}
%% @Author: Ines Abdeljaoued Tej
%  @Date:   2018-06
%% @Class:  PFE de l'ESSAI - Universite de Carthage, Tunisie.


\markboth{\MakeUppercase{Introduction}}{}%
\addcontentsline{toc}{chapter}{Introduction}%

%Welcome to \Ac{ESSAI}. ~\\
%Again, welcome to \Ac{ESSAI}. ~\\
%Your introduction goes here. ~\\

Voici une référence à l'image de la Figure \ref{fig:test} page \pageref{fig:test} et une autre vers la partie \ref{chap:2} page \pageref{chap:2}.
On peut citer un livre\, \cite{caillois1} et on précise les détails à la fin du rapport dans la partie références.
Voici une note\,\footnote{Texte de bas de page} de bas de page\footnote{J'ai bien dit bas de page}. Nous pouvons également citer l'Algorithme \ref{algo1}, la Définition \ref{def1}, le Théorème \ref{theo1} ou l'Exemple \ref{exo1}...\\

Le document est déatillé comme suit : le chapitre \ref{chap:chapterone} introduit le cadre général de ce travail. Il s'agit de présenter l'entreprise d'accueil et de détailler la problématique. Le chapitre \ref{chap:2} introduit les données ainsi que les modèles choisies. Le chapitre \ref{chap:3} donne les principaux résultats et la comparaison entre divers modèles (courbe de ROC, indice de Gini). Nous clôturons ce travail par une brève conclusion résumant le travail accompli ainsi que des perspectives qui pourraient enrichir ce travail.





\chapter{Données étudiés}%
\label{chap:chapterone}
%% @Author: Ines Abdeljaoued Tej
%  @Date:   2018-06
%% @Class:  PFE de l'ESSAI - Universite de Carthage, Tunisie.

%%%%%%%%%%%%%%%%%%%%%%%%%%%%
% SECTION                  %
%%%%%%%%%%%%%%%%%%%%%%%%%%%%
\section{Section une}
\label{chap:sectionone}

\subsection{Sub section One}

And your chapter one goes here \cite{web001,Nom2012}. \\
  Lorem ipsum dolor sit amet, consectetur adipisicing elit, sed do eiusmod
  tempor incididunt ut labore et dolore magna aliqua. Ut enim ad minim veniam, quis nostrud exercitation ullamco laboris nisi ut aliquip ex ea commodo consequat. Duis aute irure dolor in reprehenderit in voluptate velit esse \cite{Bird02nltk:the}
  cillum dolore eu fugiat nulla pariatur. Excepteur sint occaecat cupidatat non
  proident, sunt in culpa qui officia deserunt mollit anim id est laborum.

  \begin{figure}[h]%
    \center%
    \includegraphics[width=0.3\textwidth]{diamonds.pdf}
    \caption[This is a test image]{Test Image}\label{fig:test}%
  \end{figure}


\begin{table}\begin{center}
\begin{tabular}{c|c}
Entrée & Sortie \\ \hline 
A & B \\
C & D
\end{tabular}
\caption{Test Table}\end{center}
\end{table}


\subsection{Sub section Two}

  This is a second subsection\cite{gen1972}, \cite{schaeffer99}. ~\\
  Lorem ipsum dolor sit amet, consectetur adipisicing elit, sed do eiusmod
  tempor incididunt ut labore et dolore magna aliqua. Ut enim ad minim veniam,
  quis nostrud exercitation ullamco laboris nisi ut aliquip ex ea commodo
  consequat. Duis aute irure dolor in reprehenderit in voluptate velit esse
  cillum dolore eu fugiat nulla pariatur. Excepteur sint occaecat cupidatat non
  proident, sunt in culpa qui officia deserunt mollit anim id est laborum.

  \begin{description}\addtolength{\itemsep}{-0.35\baselineskip}%
    \item[\textbullet~\bfseries Menu Item] \hfill \\%
      Menu Description.~\\%
      {\textbf{Focus topics:~}\emph{Topic one, topic two, topic three, ...}}%
    %
    \item[\textbullet~\bfseries Menu Item] \hfill \\%
      Menu Description.~\\%
      {\textbf{Focus topics:~}\emph{Topic one, topic two, topic three, ...}}%
    %
    \item[\textbullet~\bfseries Menu Item] \hfill \\%
      Menu Description.~\\%
      {\textbf{Focus topics:~}\emph{Topic one, topic two, topic three, ...}}%
  \end{description}

  Also bullets such as:%
  \begin{itemize}\addtolength{\itemsep}{-0.35\baselineskip}%
    \item One%
    \item Two%
    \item Three%
    \item Four%
    \item \ldots%
  \end{itemize}%
  %
\section{powers series} \label{subsection}

\begin{equation} \label{eq:1}
\sum_{i=0}^{\infty} a_i x^i
\end{equation}

The equation \ref{eq:1} is a typical power series.

\chapter{Modèles utilisés et Applications}
\label{chap:2}
%% @Author: Ines Abdeljaoued Tej
%  @Date:   2018-06
%% @Class:  Graduation Project, ESSAI - Carthage University, Tunisia.
  

\begin{itemize}
	\item The individual \index{Entries}{entries} are indicated with a black dot, a so-called bullet.
	\item The text in the entries may be of any length.
\end{itemize}

\begin{theorem}\label{theo1}
Soit $n$ un entier naturel. Si $n$ est premier alors il n'est divisible que par 1 et par lui-même.
\end{theorem}

\begin{proof}
Here is my proof.
\end{proof}

\begin{definition}\label{def1}
Soit $A$ une courbe...
\end{definition}

Ici, il s'agit de l'utilisation de TB %\nomenclature[TB]{TB}{Très Bien} qui consiste à parler Très Bien. 
\gls{abc} et \gls{efg} sont des acronyms et des abbréviations... La méthode \gls{svm} est également couramment utilisée.

\begin{exemple}\label{exo1}
On considère le cas particulier... 
\end{exemple}

\chapter{Indicateurs de performances et Résultats}
\label{chap:3}
%% @Author: Ines Abdeljaoued Tej
%  @Date:   2018-06
%% @Class:  Graduation Project, ESSAI - Carthage University, Tunisia.


Exemple d'un algorithme : \\

\begin{algorithm}[H]
\DontPrintSemicolon
\SetAlgoLined
%\KwResult{Write here the result}
\SetKwInOut{Input}{Input}\SetKwInOut{Output}{Output}
\Input{Write here the input}
\Output{Write here the output}
\BlankLine
 
\While{While condition}{
    instructions\;
    \eIf{condition}{
        instructions1\;
        instructions2\;
    }{
        instructions3\;
    }
}
\caption{While loop with If/Else condition}\label{algo1}
\end{algorithm}



\begin{algorithm}[H]
\DontPrintSemicolon
\SetAlgoLined
%\KwResult{Write here the result}
\SetKwInOut{Input}{Entrée}\SetKwInOut{Output}{Sortie}
\Input{Write here the input}
\Output{Write here the output}
\BlankLine
 
$x\leftarrow 0$    \;
$y\leftarrow 0$    \;
\BlankLine
\ForEach{ForEach condition}{    
    
    \BlankLine
    
    \tcc{comments on code}    
    \ForEach{ForEach condition}{
        \If{If condition}{
            instruction(s) like below: \\
            increase $x$ by $1$\;
            decrease $y$ by $2$\;
        }
                
        \BlankLine
        
        \uIf{If condition}{
            instruction
        }
        \uElseIf{ElseIf condition}{
            instruction
        }
        \uElse{
            instruction
        }                
    }    
}
\caption{Nested ForEach loop with If/ElseIf/Else condition}
\end{algorithm}

\chapter*{Conclusion et Perspectives}
\label{chap:conclusion}
\markboth{\MakeUppercase{Conclusion}}{}%
\addcontentsline{toc}{chapter}{Conclusion}
  And a very interesting conclusion here\@. ~\\
  Lorem ipsum dolor sit amet, consectetur adipisicing elit, sed do eiusmod
  tempor incididunt ut labore et dolore magna aliqua. Ut enim ad minim veniam,
  quis nostrud exercitation ullamco laboris nisi ut aliquip ex ea commodo
  consequat.

\newpage
\appendix
\addcontentsline{toc}{chapter}{Annexes}
%\markboth{\MakeUppercase{Annexe}}{}

\chapter{Code R pour résoudre la problématique}
\label{chap:appendix}


\section{Pré-traitement des données}
\section{Code R pour les modèles}

 An appedix if you need it.
 
 \begin{verbatim}
 Insérer ici le code !
 \end{verbatim}

\section{Librairies utilisées}
 
  Lorem ipsum dolor sit amet, consectetur adipisicing elit, sed do eiusmod
  tempor incididunt ut labore et dolore magna aliqua. Ut enim ad minim veniam,
  quis nostrud exercitation ullamco laboris nisi ut aliquip ex ea commodo.


%%%%%%%%%%%%%%%%%%%%%%%%%%%%%%%%%%%%%%%%%%%%%%%%%%%%
% Don't touch this, it is auto generated
%%%%%%%%%%%%%%%%%%%%%%%%%%%%%%%%%%%%%%%%%%%%%%%%%%%%
\nocite{*}

%\phantomsection{}
%\addcontentsline{toc}{chapter}{Webography}
%\printbibliography[title={Webography},type=online]

%\phantomsection{}
%\addcontentsline{toc}{chapter}{Bibliography}
%\printbibliography[title={Bibliography},nottype=online]

%\printbibheading %exemple de bibliographie divisée en sections. Pour ajouter des oeuvres non citées,utiliser \nocite

%\printbibliography[keyword=pratique,heading=subbibliography,title={Théories littéraires dans les jeux vidéo}]
%\printbibliography[keyword=litteraire,heading=subbibliography,title={Narratologie et structuralisme}]

%\printbibliography[keyword=jeu,heading=subbibliography,title={\emph{Games studies}}]

\bibliographystyle{apalike}
%\bibliographystyle{plain}

\bibliography{Biblio.bib}

\cleardoublepage%

\addtocontents{toc}{\protect\setcounter{tocdepth}{3}}

\printglossaries
\printindex

This document represents a synthetic study of the work carried out in part of my second-year summer internship. I did my internship within
the international university of Rabat \ textbf {UIR} and the Directorate General of Territorial Communities (Directorate of Urban Mobility and Transport) under the Ministry of the Interior. This internship lasted both
the months of July and August of the year 2021. The main objective of this work
consists of developing a web platform for the management and monitoring of
contracts for the delegated management of urban bus transport. The project was carried out in three main stages:
\begin{itemize}
    \item[•] detailed analysis of customer needs;
    \item[•] design of the different components of the application;
    \item[•] finally the implementation of the underlined objectives.
\end{itemize}
So I had the chance to use several technologies, tools, and concepts that helped me develop this solution, including several courses that I learned during the previous year, and some tools that I had the chance to discover throughout the internship (you will find in the appendices of this document).

This report is organized as follows: Chapter I is dedicated to
the presentation of
the host organization and the client. Chapter II deals with
the analysis of the problem as well as the specification of the needs.
Chapter III presents the conceptual study, and finally chapter IV deals with the
realization part.

\textbf{\\Keywords:\\Management, Monitoring, Data, Bus, Delegated monitoring, contracts, Urban transport, Centralization, Low-code application.}



\end{document}
